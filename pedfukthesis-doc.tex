\documentclass{ltxdoc}
\usepackage[czech]{babel}
\usepackage{luavlna}
\usepackage{hyperref}
\usepackage{fontspec}
\setmainfont{TeX Gyre Schola}
\setmonofont[Scale=MatchLowercase]{Inconsolatazi4}
\usepackage{microtype}
\usepackage{hyperref}

\newcommand\pkgname{\texttt{pedfukthesis}}
\title{Balíček \pkgname}
\author{Michal Hoftich}
\date{Verze \version, \gitdate}
\begin{document}

\maketitle

\tableofcontents

\section{PDF/A}
Balíček \pkgname\ má vestavěnou podporu pro formát \texttt{PDF/A-2u},
vyžadovaný pro validaci závěrečné práce v SIS. Validitu souboru si můžete
ověřit i před nahráním do SIS pomocí nástroje
Verapdf\footnote{\url{https://verapdf.org/}}. SIS používá speciální validační
profil, který je možné získat ze stránek
\textit{Návod pro uložení kvalifikační práce}\footnote{\url{https://cuni.cz/UK-7987.html}}.

Validaci s profilem lze spusit pomocí příkazu

\begin{verbatim}
verapdf --profile UK-7987-version1-custom8.xml --format text nazevprace.pdf
\end{verbatim}

Pokud validace proběhla bez problémů, vypíše tento příkaz následující odpověď:

\begin{verbatim}
PASS nazevprace.pdf
\end{verbatim}

\subsection{Možné problémy při validaci}

Balíček \texttt{hyperref} může způsobit problémy s validitou PDF souboru. V některých případech
pomůže nahrát ho až po nahrání \pkgname. Pokud problémy přetrvají, není možné
jej použít ve verzi závěrečné práce určené pro odevzdání do SIS.


\end{document}
