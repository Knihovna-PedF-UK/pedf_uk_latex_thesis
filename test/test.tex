\documentclass[oneside,a4paper,12pt]{book}
\usepackage{iftex}
% use fontenc and inputenc only with pdflatex
\ifPDFTeX
\usepackage[T1]{fontenc}
\usepackage[utf8]{inputenc}
\fi

\usepackage[czech]{babel}


\usepackage[%
  % author={Michal Hoftich},
  % title={Ukázkový dokument},
  titletrans={Sample document},
  keywords={ukázka, test, validace},
  supervisor={PhDr. Jiří Vomáčka, Ph.D.},
  department={Katedra pedagogiky},
  programme={Specializace v pedagogice},
  subject={Testovací dokument pro účely validace},
  field={Biologie, geologie a environmentalistika se zaměřením na vzdělávání – Pedagogika}
]{../pedfukthesis}

\author{Michal Hoftich}
\title{Ukázkový titulek}

\usepackage{lipsum}
\begin{document}
\frontmatter
\TitlePage[Hello, title]

\OriginalityStatement{V Praze dne 24. 9. 2019.}

\begin{abstract}
Abstrakt v rozsahu 200 slov je stručný výtah práce. Slouží především jako pomoc
čtenáři rychle se zorientovat v dané práci, obsahuje stručnou charakteristiku
práce, její cíle a použité metody. Abstrakt obsahuje také shrnutí výsledků
práce.
\SavedKeywords
\end{abstract}

\begin{abstract}[Abstract]
An abstract of 200 words is a brief summary of the work. It serves primarily  
a reader to quickly orientate in the work, contains a brief description of the
work, its objectives and methods used. The abstract also contains a summary of
the results of the work.
\Keywords[Keywords]{test, validation}
\end{abstract}

\tableofcontents
\chapter{Úvod}
\mainmatter
\chapter{První kapitola}
\section{Sekce}
\subsection{Podsekce}
\subsubsection{Další úroveň}
\lipsum[1-3]

\end{document}
